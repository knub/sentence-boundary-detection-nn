% future work
We showed our approach to automatically detect sentence boundaries, and predict the correct punctuation marks.
Two different models were trained independently, one using lexical input and the other acoustic input.
The results of both models were merged with a late fusion.

There are many possibilties for improvement on the presented approach.
Since we did not explore a large variety of different neural network layouts, further exploration in this area could improve the results.
Especially if combined with more training data a deeper network architecture might improve the quality of the prediction.
Another approach is the usage of Long Short Term Memory (LSTM) in the neural network.

In the fusion step we decided for a late fusion approach, which only combines the predictions.
However another way to explore, is an earlier fusion, where both models and the fusion itself are trained together and the features are fused, instead of the predictions.
As for data preparation, a different representation of features in the lexical model can be examined, such as a second or third data channel or a combination similar to the fusion of the acoustic and lexical model.

The last point for minor improvements could be done with a post processing of the results.
For example, two punctuation symbols with only one or very few words between them is often not correct.
